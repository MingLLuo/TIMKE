\section{开发代码}

本章将介绍TIMKE协议实现的核心代码,关注系统的关键组件和实现细节。由于篇幅限制,仅展示最具代表性的代码片段。完整代码已开源于GitHub:\url{https://github.com/MingLLuo/TIMKE/}。

\subsection{后端核心代码}

后端代码主要包括协议核心逻辑、KEM实现、消息处理和状态管理等组件。以下是几个最关键的代码片段:

\subsubsection{KEM接口定义}

KEM接口是整个系统的核心抽象,定义了密钥封装机制的标准操作集:

\begin{minted}[breaklines]{go}
// pkg/kem/interface.go
package kem

import (
    cryptoRand "crypto/rand"
    "errors"
    "io"
)

type Parameters struct {
    Name   string
    KeyLen int
}

type PublicKey interface {
    Bytes() []byte
    Algorithm() string
}

type PrivateKey interface {
    Bytes() []byte
    Algorithm() string
    PublicKey() PublicKey
}

type KEM interface {
    // 返回KEM的参数信息
    Setup() Parameters
    
    // 生成密钥对
    GenerateKeyPair(params Parameters, rand io.Reader) (PublicKey, PrivateKey, error)
    
    // 使用公钥封装共享密钥
    Encapsulate(pk PublicKey, rand io.Reader) (ciphertext []byte, sharedSecret []byte, err error)
    
    // 使用私钥解封装共享密钥
    Decapsulate(sk PrivateKey, ciphertext []byte) ([]byte, error)
    
    // 解析二进制格式的公钥
    ParsePublicKey(data []byte) (PublicKey, error)
    
    // 解析二进制格式的私钥
    ParsePrivateKey(data []byte) (PrivateKey, error)
}

var DefaultRand = cryptoRand.Reader

var (
    ErrorKEM = errors.New("kem error")
    ErrInvalidPublicKey = errors.New("invalid public key")
    ErrInvalidPrivateKey = errors.New("invalid private key")
    ErrUnsupportedKEM = errors.New("unsupported KEM type")
)
\end{minted}

该接口设计允许协议核心逻辑与具体KEM实现分离,使系统能够无缝切换不同的KEM算法。

\subsubsection{密钥派生函数实现}

密钥派生是协议安全性的关键环节,以下是H1和H2函数的实现:

\begin{minted}[breaklines]{go}
// pkg/crypto/hash.go
package crypto

import (
    "errors"
    "TIMKE/pkg/crypto/sha3"
)

type Hash struct {
    h sha3.State
}

func NewHash() *Hash {
    return &Hash{
        h: sha3.New512(),
    }
}

func (h *Hash) Hash(data ...[]byte) ([]byte, error) {
    h.h.Reset()

    for _, d := range data {
        if _, err := h.h.Write(d); err != nil {
            return nil, err
        }
    }
    return h.h.Sum(nil), nil
}

// H1 派生临时会话密钥: K_tmp = H1(pk_S, C_1, K_1)
func H1(pkS, c1, k1 []byte) ([]byte, error) {
    if pkS == nil || c1 == nil || k1 == nil {
        return nil, errors.New("invalid input to H1")
    }

    h := NewHash()
    domain := []byte("TIMKE-H1")
    return h.Hash(domain, pkS, c1, k1)
}

// H2 派生主会话密钥: K_main = H2(pk_S, epk_C, C_1, C_2, K_1, K_2)
func H2(pkS, epkC, c1, c2, k1, k2 []byte) ([]byte, error) {
    if pkS == nil || epkC == nil || c1 == nil || c2 == nil || k1 == nil || k2 == nil {
        return nil, errors.New("invalid input to H2")
    }

    h := NewHash()
    domain := []byte("TIMKE-H2")
    return h.Hash(domain, pkS, epkC, c1, c2, k1, k2)
}
\end{minted}

这些函数使用SHA3-512哈希函数,并通过域分隔标识("TIMKE-H1"和"TIMKE-H2")确保不同上下文的哈希值不会冲突。

\subsubsection{客户端实现}

客户端负责启动密钥交换流程,生成ClientHello消息并处理服务器响应:

\begin{minted}[breaklines]{go}
// pkg/protocol/client.go
package protocol

import (
    "errors"
    "fmt"
    "io"

    "TIMKE/pkg/crypto"
    "TIMKE/pkg/kem"
)

type Client struct {
    config  *Config
    state   SessionState
    options *SessionOptions
    rand    io.Reader

    ephemeralPublicKey  kem.PublicKey
    ephemeralPrivateKey kem.PrivateKey
    ciphertext1         []byte
    sharedSecret1       []byte // K_1
    tempKey             []byte // K_tmp
    ciphertext2         []byte
    sharedSecret2       []byte // K_2
    sessionKey          []byte // K_main
}

func NewClient(config *Config, options *SessionOptions) (*Client, error) {
    if config == nil {
        config = DefaultConfig()
    }

    if options == nil || options.ServerPublicKey == nil {
        return nil, errors.New("server public key is required")
    }

    return &Client{
        config:  config,
        state:   StateInitial,
        options: options,
        rand:    kem.DefaultRand,
    }, nil
}

func (c *Client) GenerateClientHello(zeroRTTData []byte) (*ClientHello, error) {
    if c.state != StateInitial {
        return nil, errors.New("client not in initial state")
    }

    // 1. 生成临时密钥对
    params := c.config.KEM2.Setup()
    epk, esk, err := c.config.KEM2.GenerateKeyPair(params, c.rand)
    if err != nil {
        c.state = StateFailed
        return nil, fmt.Errorf("failed to generate ephemeral key pair: %w", err)
    }
    c.ephemeralPublicKey = epk
    c.ephemeralPrivateKey = esk

    // 2. 使用服务器公钥封装KEM1
    c.ciphertext1, c.sharedSecret1, err = c.config.KEM1.Encapsulate(c.options.ServerPublicKey, c.rand)
    if err != nil {
        c.state = StateFailed
        return nil, fmt.Errorf("failed to encapsulate KEM1: %w", err)
    }

    // 3. 派生临时会话密钥: K_tmp = H1(server_pk, C₁, K_1)
    c.tempKey, err = crypto.H1(
        c.options.ServerPublicKey.Bytes(),
        c.ciphertext1,
        c.sharedSecret1,
    )
    if err != nil {
        c.state = StateFailed
        return nil, fmt.Errorf("failed to derive temp key: %w", err)
    }

    // 4. 加密0-RTT数据
    var encryptedPayload []byte
    if zeroRTTData != nil {
        encryptedPayload, err = c.config.SymmetricEncryption.Encrypt(c.tempKey, zeroRTTData)
        if err != nil {
            c.state = StateFailed
            return nil, fmt.Errorf("failed to encrypt 0-RTT data: %w", err)
        }
    }

    // 5. 构造ClientHello消息
    clientHello := &ClientHello{
        EphemeralPublicKey: c.ephemeralPublicKey.Bytes(),
        Ciphertext1:        c.ciphertext1,
        EncryptedPayload:   encryptedPayload,
        KEM1Type:           c.config.KEM1.Setup().Name,
        KEM2Type:           c.config.KEM2.Setup().Name,
    }

    c.state = StateAwaitingServerResponse
    return clientHello, nil
}

func (c *Client) ProcessServerResponse(response *ServerResponse) ([]byte, error) {
    if c.state != StateAwaitingServerResponse {
        return nil, errors.New("client not waiting for server response")
    }

    if response == nil {
        c.state = StateFailed
        return nil, errors.New("nil server response")
    }

    // 处理服务器响应,解封装KEM2密文
    c.ciphertext2 = response.Ciphertext2
    sharedSecret2, err := c.config.KEM2.Decapsulate(c.ephemeralPrivateKey, c.ciphertext2)
    if err != nil {
        c.state = StateFailed
        return nil, fmt.Errorf("failed to decapsulate KEM2: %w", err)
    }
    c.sharedSecret2 = sharedSecret2

    // 派生主会话密钥
    c.sessionKey, err = crypto.H2(
        c.options.ServerPublicKey.Bytes(),
        c.ephemeralPublicKey.Bytes(),
        c.ciphertext1,
        c.ciphertext2,
        c.sharedSecret1,
        c.sharedSecret2,
    )
    if err != nil {
        c.state = StateFailed
        return nil, fmt.Errorf("failed to derive session key: %w", err)
    }

    // 解密服务器有效载荷(如果有)
    var plaintext []byte
    if len(response.EncryptedPayload) > 0 {
        plaintext, err = c.config.SymmetricEncryption.Decrypt(c.sessionKey, response.EncryptedPayload)
        if err != nil {
            c.state = StateFailed
            return nil, fmt.Errorf("failed to decrypt server payload: %w", err)
        }
    }

    c.state = StateEstablished
    return plaintext, nil
}

// 其他方法...
\end{minted}

客户端实现了协议的两个阶段:生成ClientHello消息(包含临时密钥生成、KEM1封装和0-RTT数据加密)和处理服务器响应(包括KEM2解封装和主会话密钥派生)。

\subsubsection{服务器实现}

服务器负责处理客户端请求,验证身份并建立共享密钥:

\begin{minted}[breaklines]{go}
// pkg/protocol/server.go
package protocol

import (
    "errors"
    "fmt"
    "io"

    "TIMKE/pkg/crypto"
    "TIMKE/pkg/kem"
)

type Server struct {
    config  *Config
    state   SessionState
    options *SessionOptions
    rand    io.Reader

    ephemeralClientPubKey kem.PublicKey
    ciphertext1           []byte
    sharedSecret1         []byte // K_1
    tempKey               []byte // K_tmp
    ciphertext2           []byte
    sharedSecret2         []byte // K_2
    sessionKey            []byte // K_main

    dynamicKEM1 kem.KEM
    dynamicKEM2 kem.KEM
}

func NewServer(config *Config, options *SessionOptions) (*Server, error) {
    if config == nil {
        config = DefaultConfig()
    }

    if options == nil || options.ServerPrivateKey == nil {
        return nil, errors.New("server private key is required")
    }

    return &Server{
        config:  config,
        state:   StateInitial,
        options: options,
        rand:    kem.DefaultRand,
    }, nil
}

func (s *Server) ProcessClientHello(clientHello *ClientHello) ([]byte, error) {
    if s.state != StateInitial {
        return nil, errors.New("server not in initial state")
    }

    if clientHello == nil {
        s.state = StateFailed
        return nil, errors.New("nil client hello")
    }

    // 处理KEM类型协商(如果提供)
    var err error
    if clientHello.KEM1Type != "" && clientHello.KEM2Type != "" {
        s.dynamicKEM1, err = SelectKEM(clientHello.KEM1Type)
        if err != nil {
            s.dynamicKEM1 = DefaultKEM1()
        }

        s.dynamicKEM2, err = SelectKEM(clientHello.KEM2Type)
        if err != nil {
            s.dynamicKEM2 = DefaultKEM2()
        }
    } else {
        s.dynamicKEM1 = DefaultKEM1()
        s.dynamicKEM2 = DefaultKEM2()
    }

    // 1. 解析客户端临时公钥
    s.ephemeralClientPubKey, err = s.dynamicKEM2.ParsePublicKey(clientHello.EphemeralPublicKey)
    if err != nil {
        s.state = StateFailed
        return nil, fmt.Errorf("failed to parse ephemeral public key: %w", err)
    }

    s.ciphertext1 = clientHello.Ciphertext1

    // 2. 使用服务器私钥解封装KEM1密文
    s.sharedSecret1, err = s.dynamicKEM1.Decapsulate(s.options.ServerPrivateKey, s.ciphertext1)
    if err != nil {
        s.state = StateFailed
        return nil, fmt.Errorf("failed to decapsulate KEM1: %w", err)
    }

    // 3. 派生临时会话密钥
    serverPubKey := s.options.ServerPrivateKey.PublicKey()
    s.tempKey, err = crypto.H1(
        serverPubKey.Bytes(),
        s.ciphertext1,
        s.sharedSecret1,
    )
    if err != nil {
        s.state = StateFailed
        return nil, fmt.Errorf("failed to derive temp key: %w", err)
    }

    // 4. 解密0-RTT数据(如果有)
    var zeroRTTData []byte
    if len(clientHello.EncryptedPayload) > 0 {
        zeroRTTData, err = s.config.SymmetricEncryption.Decrypt(s.tempKey, clientHello.EncryptedPayload)
        if err != nil {
            s.state = StateFailed
            return nil, fmt.Errorf("failed to decrypt 0-RTT data: %w", err)
        }
    }

    return zeroRTTData, nil
}

func (s *Server) GenerateServerResponse(payload []byte) (*ServerResponse, error) {
    if s.ephemeralClientPubKey == nil || s.sharedSecret1 == nil {
        return nil, errors.New("client hello not processed")
    }

    // 1. 使用客户端临时公钥封装KEM2
    var err error
    s.ciphertext2, s.sharedSecret2, err = s.dynamicKEM2.Encapsulate(s.ephemeralClientPubKey, s.rand)
    if err != nil {
        s.state = StateFailed
        return nil, fmt.Errorf("failed to encapsulate KEM2: %w", err)
    }

    // 2. 派生主会话密钥
    serverPubKey := s.options.ServerPrivateKey.PublicKey()
    s.sessionKey, err = crypto.H2(
        serverPubKey.Bytes(),
        s.ephemeralClientPubKey.Bytes(),
        s.ciphertext1,
        s.ciphertext2,
        s.sharedSecret1,
        s.sharedSecret2,
    )
    if err != nil {
        s.state = StateFailed
        return nil, fmt.Errorf("failed to derive session key: %w", err)
    }

    // 3. 加密有效载荷(如果有)
    var encryptedPayload []byte
    if payload != nil {
        encryptedPayload, err = s.config.SymmetricEncryption.Encrypt(s.sessionKey, payload)
        if err != nil {
            s.state = StateFailed
            return nil, fmt.Errorf("failed to encrypt payload: %w", err)
        }
    }

    // 4. 构造服务器响应
    serverResponse := &ServerResponse{
        Ciphertext2:      s.ciphertext2,
        EncryptedPayload: encryptedPayload,
    }

    s.state = StateEstablished
    return serverResponse, nil
}

// 其他方法...
\end{minted}

服务器实现了协议的两个主要功能:处理ClientHello消息(包括解析临时公钥、解封装KEM1密文和解密0-RTT数据)和生成ServerResponse(包括KEM2封装和有效载荷加密)。

\subsubsection{消息序列化}

协议消息的序列化是确保通信正确性的关键组件:

\begin{minted}[breaklines]{go}
// pkg/protocol/serializer.go
package protocol

import (
    "encoding/binary"
    "errors"
    "fmt"
)

// 消息结构定义
type ClientHello struct {
    EphemeralPublicKey []byte
    Ciphertext1        []byte
    EncryptedPayload   []byte
    KEM1Type           string
    KEM2Type           string
}

type ServerResponse struct {
    Ciphertext2      []byte
    EncryptedPayload []byte
}

// DefaultSerializer 实现消息序列化接口
type DefaultSerializer struct{}

// MarshalClientHello 序列化ClientHello消息
func (s *DefaultSerializer) MarshalClientHello(ch *ClientHello) ([]byte, error) {
    if ch == nil {
        return nil, errors.New("cannot marshal nil ClientHello")
    }

    // 预分配合理大小的缓冲区
    estimatedSize := 4 + len(ch.EphemeralPublicKey) +
        4 + len(ch.Ciphertext1) +
        4 + len(ch.EncryptedPayload) +
        4 + len([]byte(ch.KEM1Type)) +
        4 + len([]byte(ch.KEM2Type))

    result := make([]byte, 0, estimatedSize)

    // 使用长度前缀编码每个字段
    result = writeLengthPrefixedBytes(result, ch.EphemeralPublicKey)
    result = writeLengthPrefixedBytes(result, ch.Ciphertext1)
    result = writeLengthPrefixedBytes(result, ch.EncryptedPayload)
    result = writeLengthPrefixedBytes(result, []byte(ch.KEM1Type))
    result = writeLengthPrefixedBytes(result, []byte(ch.KEM2Type))

    return result, nil
}

// UnmarshalClientHello 反序列化ClientHello消息
func (s *DefaultSerializer) UnmarshalClientHello(data []byte) (*ClientHello, error) {
    if len(data) < 2 {
        return nil, ErrInvalidMessage
    }

    ch := &ClientHello{}
    offset := 0
    var err error

    // 读取各字段
    ch.EphemeralPublicKey, offset, err = readLengthPrefixedBytes(data, offset)
    if err != nil {
        return nil, err
    }

    ch.Ciphertext1, offset, err = readLengthPrefixedBytes(data, offset)
    if err != nil {
        return nil, err
    }

    ch.EncryptedPayload, offset, err = readLengthPrefixedBytes(data, offset)
    if err != nil {
        return nil, err
    }

    kem1TypeBytes, offset, err := readLengthPrefixedBytes(data, offset)
    if err != nil {
        return nil, err
    }
    ch.KEM1Type = string(kem1TypeBytes)

    kem2TypeBytes, offset, err := readLengthPrefixedBytes(data, offset)
    if err != nil {
        return nil, err
    }
    ch.KEM2Type = string(kem2TypeBytes)

    // 检查是否已读取全部数据
    if offset != len(data) {
        return ch, errors.New("extra data after message")
    }

    return ch, nil
}

// 辅助函数
func readLengthPrefixedBytes(data []byte, offset int) ([]byte, int, error) {
    // 检查是否可以读取长度字段(4字节)
    if offset+4 > len(data) {
        return nil, offset, ErrBufferTooShort
    }

    // 读取长度
    length := binary.BigEndian.Uint32(data[offset:])
    offset += 4

    // 检查是否可以读取数据
    if offset+int(length) > len(data) {
        return nil, offset, ErrBufferTooShort
    }

    // 提取数据
    result := make([]byte, length)
    copy(result, data[offset:offset+int(length)])
    offset += int(length)

    return result, offset, nil
}

func writeLengthPrefixedBytes(result []byte, data []byte) []byte {
    // 分配4字节存储长度
    lengthBytes := make([]byte, 4)
    binary.BigEndian.PutUint32(lengthBytes, uint32(len(data)))

    // 追加长度和数据
    result = append(result, lengthBytes...)
    result = append(result, data...)

    return result
}

// 其他序列化方法...
\end{minted}
消息序列化代码定义了ClientHello和ServerResponse消息的格式,并提供了序列化和反序列化方法,使用长度前缀编码确保消息的正确解析。

\subsection{前端核心代码}

TIMKE协议实现主要为命令行应用,包含客户端和服务器两个前端程序,以及演示脚本。

\subsubsection{客户端程序}

客户端程序提供命令行接口,支持与服务器建立TIMKE会话:

\begin{minted}[breaklines]{go}
// cmd/client/main.go
package main

import (
    "bufio"
    "encoding/binary"
    "encoding/hex"
    "flag"
    "fmt"
    "io"
    "log"
    "net"
    "os"
    "strings"
    "time"

    "TIMKE/pkg/kem"
    "TIMKE/pkg/protocol"
)

// 颜色常量定义
const (
    colorReset  = "\033[0m"
    colorRed    = "\033[31m"
    colorGreen  = "\033[32m"
    colorYellow = "\033[33m"
    colorBlue   = "\033[34m"
    colorPurple = "\033[35m"
    colorCyan   = "\033[36m"
)

func main() {
    // 命令行参数定义
    var (
        host          = flag.String("host", "localhost", "服务器主机名或IP")
        port          = flag.Int("port", 8443, "服务器端口")
        serverPubKey  = flag.String("server-key", "", "服务器公钥(十六进制格式)")
        serverKeyFile = flag.String("server-key-file", "", "包含服务器公钥的文件路径")
        kem1Type      = flag.String("kem1", "ML-KEM-768", "服务器密钥KEM类型")
        kem2Type      = flag.String("kem2", "ML-KEM-768", "临时密钥KEM类型")
        zeroRTTMsg    = flag.String("0rtt", "Hello from TIMKE client! This is 0-RTT data.", "0-RTT消息(空字符串禁用)")
        interactive   = flag.Bool("i", false, "交互模式(密钥交换后发送/接收消息)")
        verbose       = flag.Bool("v", false, "显示详细输出")
    )
    flag.Parse()

    logger := log.New(os.Stdout, "", 0)

    // 打印标题
    printBanner(logger)

    // 列出可用的KEM算法
    logger.Printf("%s可用的KEM算法:%s\n", colorYellow, colorReset)
    for _, k := range kem.ListKEMs() {
        logger.Printf("  - %s\n", k)
    }
    logger.Println()

    // 获取服务器公钥
    var serverPublicKeyBytes []byte
    var err error

    if *serverKeyFile != "" {
        // 从文件加载
        serverPublicKeyBytes, err = os.ReadFile(*serverKeyFile)
        if err != nil {
            logger.Fatalf("%s读取服务器密钥文件错误: %s%s\n", colorRed, err, colorReset)
        }
    } else if *serverPubKey != "" {
        // 解析十六进制字符串
        serverPublicKeyBytes, err = hex.DecodeString(*serverPubKey)
        if err != nil {
            logger.Fatalf("%s解码服务器公钥错误: %s%s\n", colorRed, err, colorReset)
        }
    } else {
        logger.Fatalf("%s错误: 必须提供--server-key或--server-key-file参数%s\n", colorRed, colorReset)
    }

    // 获取KEM实例
    kem1, err := kem.GetKEM(*kem1Type)
    if err != nil {
        logger.Fatalf("%s错误: KEM1类型'%s'未找到: %s%s\n", colorRed, *kem1Type, err, colorReset)
    }
    kem2, err := kem.GetKEM(*kem2Type)
    if err != nil {
        logger.Fatalf("%s错误: KEM2类型'%s'未找到: %s%s\n", colorRed, *kem2Type, err, colorReset)
    }

    // 解析服务器公钥
    serverPublicKey, err := kem1.ParsePublicKey(serverPublicKeyBytes)
    if err != nil {
        logger.Fatalf("%s解析服务器公钥错误: %s%s\n", colorRed, err, colorReset)
    }

    logger.Printf("%s使用服务器密钥: %s%s\n", colorGreen, serverPublicKey.Algorithm(), colorReset)

    // 创建客户端配置
    config := &protocol.Config{
        KEM1:                kem1,
        KEM2:                kem2,
        SymmetricEncryption: protocol.DefaultConfig().SymmetricEncryption,
    }

    // 创建客户端选项
    options := protocol.NewSessionOptions().WithServerPublicKey(serverPublicKey)

    // 创建客户端
    client, err := protocol.NewClient(config, options)
    if err != nil {
        logger.Fatalf("%s创建客户端错误: %s%s\n", colorRed, err, colorReset)
    }

    // 连接服务器
    serverAddr := fmt.Sprintf("%s:%d", *host, *port)
    logger.Printf("%s连接到%s...%s\n", colorYellow, serverAddr, colorReset)
    conn, err := net.Dial("tcp", serverAddr)
    if err != nil {
        logger.Fatalf("%s连接服务器错误: %s%s\n", colorRed, err, colorReset)
    }
    defer conn.Close()
    logger.Printf("%s已连接到%s%s\n", colorGreen, serverAddr, colorReset)

    // 准备0-RTT数据
    var zeroRTTData []byte
    if *zeroRTTMsg != "" {
        zeroRTTData = []byte(*zeroRTTMsg)
    }

    // 开始协议 - 生成ClientHello
    startTime := time.Now()
    logger.Printf("%s生成ClientHello...%s\n", colorCyan, colorReset)
    clientHello, err := client.GenerateClientHello(zeroRTTData)
    if err != nil {
        logger.Fatalf("%s生成ClientHello错误: %s%s\n", colorRed, err, colorReset)
    }

    // 协议可视化 - 第一阶段
    if *verbose {
        logger.Printf("%s---------- 协议第一阶段 ----------%s\n", colorPurple, colorReset)
        logger.Printf("KEM1类型: %s\n", clientHello.KEM1Type)
        logger.Printf("KEM2类型: %s\n", clientHello.KEM2Type)
        logger.Printf("临时公钥长度: %d字节\n", len(clientHello.EphemeralPublicKey))
        logger.Printf("密文1长度: %d字节\n", len(clientHello.Ciphertext1))
        if zeroRTTData != nil {
            logger.Printf("0-RTT数据: %s\n", *zeroRTTMsg)
            logger.Printf("加密载荷长度: %d字节\n", len(clientHello.EncryptedPayload))
        } else {
            logger.Printf("无0-RTT数据\n")
        }
    }

    // 序列化并发送ClientHello
    serializer := &protocol.DefaultSerializer{}
    clientHelloBytes, err := serializer.MarshalClientHello(clientHello)
    if err != nil {
        logger.Fatalf("%s序列化ClientHello错误: %s%s\n", colorRed, err, colorReset)
    }

    // 首先发送长度(4字节,32位)
    lenBuf := make([]byte, 4)
    binary.BigEndian.PutUint32(lenBuf, uint32(len(clientHelloBytes)))
    if _, err := conn.Write(lenBuf); err != nil {
        logger.Fatalf("%s发送ClientHello长度错误: %s%s\n", colorRed, err, colorReset)
    }

    // 然后发送实际的ClientHello
    if _, err := conn.Write(clientHelloBytes); err != nil {
        logger.Fatalf("%s发送ClientHello错误: %s%s\n", colorRed, err, colorReset)
    }

    logger.Printf("%s已发送ClientHello (%d字节)%s\n", colorGreen, len(clientHelloBytes), colorReset)
    if zeroRTTData != nil {
        logger.Printf("%s已发送0-RTT数据: \"%s\"%s\n", colorPurple, *zeroRTTMsg, colorReset)
    }

    // 读取服务器响应
    logger.Printf("%s等待服务器响应...%s\n", colorCyan, colorReset)

    // 读取消息长度
    if _, err := io.ReadFull(conn, lenBuf); err != nil {
        logger.Fatalf("%s读取服务器响应长度错误: %s%s\n", colorRed, err, colorReset)
    }

    messageLen := int(binary.BigEndian.Uint32(lenBuf))
    messageBuf := make([]byte, messageLen)
    if _, err := io.ReadFull(conn, messageBuf); err != nil {
        logger.Fatalf("%s读取服务器响应错误: %s%s\n", colorRed, err, colorReset)
    }

    // 反序列化服务器响应
    serverResponse, err := serializer.UnmarshalServerResponse(messageBuf)
    if err != nil {
        logger.Fatalf("%s反序列化服务器响应错误: %s%s\n", colorRed, err, colorReset)
    }

    logger.Printf("%s收到服务器响应 (%d字节)%s\n", colorGreen, len(messageBuf), colorReset)

    // 协议可视化 - 第二阶段
    if *verbose {
        logger.Printf("%s---------- 协议第二阶段 ----------%s\n", colorPurple, colorReset)
        logger.Printf("密文2长度: %d字节\n", len(serverResponse.Ciphertext2))
        logger.Printf("加密载荷长度: %d字节\n", len(serverResponse.EncryptedPayload))
    }

    // 处理服务器响应
    serverData, err := client.ProcessServerResponse(serverResponse)
    if err != nil {
        logger.Fatalf("%s处理服务器响应错误: %s%s\n", colorRed, err, colorReset)
    }

    elapsedTime := time.Since(startTime)
    sessionKey := client.GetSessionKey()

    // 会话已建立!
    logger.Printf("%s会话已建立! 协议完成时间: %v%s\n", colorGreen, elapsedTime, colorReset)
    if *verbose && len(sessionKey) > 0 {
        logger.Printf("%s会话密钥(前8字节): %x%s\n", colorPurple, sessionKey[:min(8, len(sessionKey))], colorReset)
    }

    // 显示服务器数据
    if len(serverData) > 0 {
        logger.Printf("%s服务器数据: %s%s\n", colorPurple, string(serverData), colorReset)
    }

    // 协议完成!
    logger.Printf("%sTIMKE协议成功完成!%s\n", colorBlue, colorReset)

    // 交互模式
    if *interactive {
        // 实现交互式消息收发...
    }
}

// 其他辅助函数...
\end{minted}

客户端程序提供了用户友好的命令行接口,支持多种参数配置,并通过颜色编码输出增强用户体验。程序实现了完整的TIMKE协议交互流程,包括建立连接、发送ClientHello、处理服务器响应和显示会话状态。

\subsubsection{演示脚本}

演示脚本提供了用户友好的交互式界面,整合了协议的所有功能:

\begin{minted}[breaklines]{bash}
#!/bin/bash

# 颜色定义
RED='\033[0;31m'
GREEN='\033[0;32m'
YELLOW='\033[0;33m'
BLUE='\033[0;34m'
CYAN='\033[0;36m'
NC='\033[0m' # No Color

# 默认设置
KEM1_TYPE="ML-KEM-768"
KEM2_TYPE="ML-KEM-768"
PORT=8443
TEMP_DIR="$(pwd)/.temp"
SERVER_KEY="${TEMP_DIR}/server-key.pem"
CLIENT_ZERO_RTT="Hello from TIMKE client! This is 0-RTT data."
SERVER_PID_FILE="${TEMP_DIR}/server.pid"

# 标题
...

# 解析命令行参数
while [[ $# -gt 0 ]]; do
    case $1 in
        --kem1)
            KEM1_TYPE="$2"
            shift 2
            ;;
        --kem2)
            KEM2_TYPE="$2"
            shift 2
            ;;
        --port)
            PORT="$2"
            shift 2
            ;;
        --help)
            print_banner
            echo -e "${GREEN}TIMKE演示运行脚本${NC}"
            echo -e "用法: $0 [选项]"
            echo -e ""
            echo -e "选项:"
            echo -e "  --kem1 TYPE     要使用的KEM1算法 (默认: ML-KEM-768)"
            echo -e "  --kem2 TYPE     要使用的KEM2算法 (默认: ML-KEM-768)"
            echo -e "  --port PORT     服务器端口 (默认: 8443)"
            echo -e "  --help          显示帮助信息"
            exit 0
            ;;
        *)
            echo -e "${RED}未知选项: $1${NC}"
            exit 1
            ;;
    esac
done

# 退出时清理
cleanup() {
    echo -e "${YELLOW}清理临时文件...${NC}"
    
    # 检查服务器是否运行,如果是则停止
    if [ -f "${SERVER_PID_FILE}" ]; then
        SERVER_PID=$(cat "${SERVER_PID_FILE}")
        if kill -0 "${SERVER_PID}" 2>/dev/null; then
            echo -e "${YELLOW}停止TIMKE服务器 (PID: ${SERVER_PID})...${NC}"
            kill "${SERVER_PID}" 2>/dev/null || true
        fi
    fi
}

# 生成服务器密钥
generate_keys() {
    echo -e "${YELLOW}使用kem1.(${KEM1_TYPE})生成服务器密钥...${NC}"

    # 移动到项目根目录
    cd "$(dirname "$0")/.." || exit 1
    # 检查临时目录是否存在
    if [[ ! -d "${TEMP_DIR}" ]]; then
        mkdir -p "${TEMP_DIR}"
    fi

    go run ./cmd/server/main.go --genkey "${SERVER_KEY}" --kem1 "${KEM1_TYPE}" --kem2 "${KEM2_TYPE}"
    
    if [[ ! -f "${SERVER_KEY}" || ! -f "${SERVER_KEY}.pub" ]]; then
        echo -e "${RED}生成服务器密钥失败.${NC}"
        return 1
    fi
    
    echo -e "${GREEN}服务器密钥已生成:${NC}"
    echo -e "  私钥: ${SERVER_KEY}"
    echo -e "  公钥: ${SERVER_KEY}.pub"
    echo -e "\n"
    
    return 0
}

# 启动服务器
start_server() {
    # 移动到项目根目录
    cd "$(dirname "$0")/.." || exit 1

    # 检查临时目录是否存在
    if [[ ! -d "${TEMP_DIR}" ]]; then
        mkdir -p "${TEMP_DIR}"
    fi

    # 检查服务器是否已经运行
    if [ -f "${SERVER_PID_FILE}" ]; then
        local old_pid=$(cat "${SERVER_PID_FILE}")
        if kill -0 "${old_pid}" 2>/dev/null; then
            echo -e "${YELLOW}服务器已在运行中,PID: ${old_pid}${NC}"
            return 0
        fi
    fi

    # 检查密钥是否存在
    if [[ ! -f "${SERVER_KEY}" ]]; then
        echo -e "${RED}未找到服务器密钥。请先生成密钥.${NC}"
        return 1
    fi

    echo -e "${YELLOW}在端口 ${PORT} 上启动TIMKE服务器...${NC}"
    
    # 后台启动服务器
    TIMESTAMP=$(date +"%Y-%m-%d_%H-%M-%S")
    echo "TIMKE服务器日志文件" > "${TEMP_DIR}/server.log"
    echo "时间戳: ${TIMESTAMP}" >> "${TEMP_DIR}/server.log"
    go run ./cmd/server/main.go --key "${SERVER_KEY}" --port "${PORT}" --kem1 "${KEM1_TYPE}" --kem2 "${KEM2_TYPE}" --v > "${TEMP_DIR}/server.log" 2>&1 &
    local server_pid=$!
    
    # 保存PID用于后续清理
    echo "${server_pid}" > "${SERVER_PID_FILE}"
    
    # 等待片刻让服务器启动
    sleep 2
    
    # 检查服务器是否运行
    if ! kill -0 "${server_pid}" 2>/dev/null; then
        echo -e "${RED}服务器启动失败。查看服务器日志: ${TEMP_DIR}/server.log${NC}"
        cat "${TEMP_DIR}/server.log"
        return 1
    fi
    
    echo -e "${GREEN}服务器已启动,PID: ${server_pid}${NC}"
    echo -e "${BLUE}服务器日志路径: ${TEMP_DIR}/server.log${NC}"
    echo -e "${YELLOW}服务器日志尾部:${NC}"
    tail -n 10 "${TEMP_DIR}/server.log"
    echo
    
    return 0
}

# 其他功能实现...

# 注册清理函数
trap cleanup EXIT

# 主菜单
show_menu() {
    echo -e "${GREEN}TIMKE演示选项:${NC}"
    echo -e "  ${YELLOW}1)${NC} 生成服务器密钥"
    echo -e "  ${YELLOW}2)${NC} 启动服务器"
    echo -e "  ${YELLOW}3)${NC} 运行携带0-RTT数据的客户端"
    echo -e "  ${YELLOW}4)${NC} 运行交互式客户端"
    echo -e "  ${YELLOW}5)${NC} 显示服务器日志"
    echo -e "  ${YELLOW}6)${NC} 停止服务器"
    echo -e "  ${YELLOW}7)${NC} 退出"
    echo
    echo -e "${BLUE}使用KEM1: ${KEM1_TYPE}, KEM2: ${KEM2_TYPE}, 端口: ${PORT}${NC}"
    echo
    
    # 检查服务器状态
    if [ -f "${SERVER_PID_FILE}" ]; then
        local pid=$(cat "${SERVER_PID_FILE}")
        if kill -0 "${pid}" 2>/dev/null; then
            echo -e "${GREEN}服务器正在运行,PID: ${pid}${NC}"
        else
            echo -e "${RED}服务器未运行(过期的PID文件)${NC}"
            rm -f "${SERVER_PID_FILE}"
        fi
    else
        echo -e "${YELLOW}服务器未运行${NC}"
    fi
    echo
}

# 主循环
print_banner

while true; do
    show_menu
    echo -n "输入您的选择 [1-7]: "
    read -r choice
    
    case $choice in
        1)
            generate_keys
            echo -e "${YELLOW}按Enter继续...${NC}"
            read -r
            ;;
        2)
            start_server
            echo -e "${YELLOW}按Enter继续...${NC}"
            read -r
            ;;
        3)
            run_client_0rtt
            echo -e "${YELLOW}按Enter继续...${NC}"
            read -r
            ;;
        4)
            run_client_interactive
            ;;
        5)
            show_server_log
            echo -e "${YELLOW}按Enter继续...${NC}"
            read -r
            ;;
        6)
            stop_server
            echo -e "${YELLOW}按Enter继续...${NC}"
            read -r
            ;;
        7)
            echo -e "${GREEN}退出TIMKE演示.${NC}"
            exit 0
            ;;
        *)
            echo -e "${RED}无效选择。请输入1到7之间的数字.${NC}"
            sleep 1
            ;;
    esac
    
    clear
    print_banner
done
\end{minted}

演示脚本提供了用户友好的菜单界面,集成了服务器密钥生成、服务器启动、客户端连接等所有核心功能,使用户能够轻松体验TIMKE协议的完整流程,是展示系统功能的重要工具。

通过以上代码示例,展示了TIMKE协议实现的核心组件和关键逻辑。完整代码实现了更多功能和优化,可在GitHub代码库中查看。