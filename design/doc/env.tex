\section{开发工具与运行环境}

\subsection{开发工具}

本项目采用现代软件工程方法进行开发,使用了以下主要开发工具:

\begin{itemize}
    \item \textbf{编程语言}:主要使用Go语言(版本1.23.4)进行开发,该语言具有以下优势:
    \begin{itemize}
        \item 强大的并发处理能力,适合高性能密码学应用
        \item 完善的标准库,包括丰富的密码学原语支持
        \item 强类型系统和内存安全特性,减少常见安全漏洞
        \item 跨平台兼容性,支持多种操作系统和硬件架构
    \end{itemize}
    
    \item \textbf{开发环境}:
    \begin{itemize}
        \item Visual Studio Code:主要代码编辑器,配置了Go语言插件集成
        \item GoLand:JetBrains公司的Go专用IDE,提供高级调试和代码分析功能
        \item Git:版本控制系统,用于代码管理和协作
        \item GitHub:代码托管平台,用于项目公开和分享
    \end{itemize}
    
    \item \textbf{测试工具}:
    \begin{itemize}
        \item Go内置测试框架:用于单元测试和集成测试
        \item Go Benchmark:用于性能测试和比较
        \item Bash脚本:用于自动化测试和演示
    \end{itemize}
    
    \item \textbf{文档工具}:
    \begin{itemize}
        \item Markdown:编写项目文档和说明
        \item LaTeX:编写学术论文和技术报告
        \item Draw.io:创建系统架构和流程图
    \end{itemize}
\end{itemize}

\subsection{运行环境}

TIMKE协议实现设计为可在多种环境下运行,以下是推荐的运行环境配置:

\subsubsection{硬件要求}

\begin{itemize}
    \item \textbf{CPU}:现代多核处理器(推荐4核心或以上)
    \item \textbf{内存}:
    \begin{itemize}
        \item 基础运行:最低4GB RAM
        \item OW-ChCCA实现:推荐16GB以上RAM
        \item ML-KEM实现:4GB RAM即可满足需求
    \end{itemize}
    \item \textbf{存储}:至少200MB可用空间(源代码、编译产物和演示数据)
    \item \textbf{网络}:支持TCP/IP网络连接(用于客户端-服务器通信)
\end{itemize}

\subsubsection{软件环境}

\begin{itemize}
    \item \textbf{操作系统}:
    \begin{itemize}
        \item Linux(Ubuntu 20.04+,CentOS 8+等)
        \item macOS(10.15+)
        \item Windows 10/11(使用Windows Subsystem for Linux或原生Go环境)
    \end{itemize}
    
    \item \textbf{运行时依赖}:
    \begin{itemize}
        \item Go语言运行时(1.18+,推荐1.23+)
        \item Bash 4.0+(用于演示脚本)
    \end{itemize}
    
    \item \textbf{库依赖}:
    \begin{itemize}
        \item Cloudflare的CIRCL库:提供ML-KEM等后量子密码学实现
        \item Tuneinsight的Lattigo库:提供格密码学相关功能
        \item 标准Go库中的crypto包:提供基础密码学功能
    \end{itemize}
\end{itemize}

值得注意的是,不同KEM算法配置对系统资源的要求差异较大。使用ML-KEM配置时,系统资源需求较低,适合包括移动设备在内的各类环境;而使用自实现的OW-ChCCA KEM时,特别是高安全级别配置,需要更多内存和处理能力。

\subsubsection{开发环境配置}

本项目的开发和测试主要在以下环境中进行:

\begin{itemize}
    \item \textbf{操作系统}:macOS Sequoia 15
    \item \textbf{处理器}:Apple M1 Pro(8核,6性能/2能效)
    \item \textbf{内存}:16GB RAM
    \item \textbf{Go版本}:1.23.4
    \item \textbf{编辑器}:Visual Studio Code / GoLand
    \item \textbf{终端}:iTerm2 + Bash
\end{itemize}

\subsection{安装方法}

以下为TIMKE协议实现的安装步骤,适用于不同操作系统环境:

\subsubsection{前提条件}

确保系统已安装:
\begin{itemize}
    \item Go语言环境(1.18或更高版本)
    \item Git版本控制工具
    \item Bash shell(Linux/macOS自带,Windows需安装Git Bash或WSL)
\end{itemize}

\subsubsection{获取源代码}

通过Git克隆项目仓库:
\begin{minted}[breaklines]{bash}
# 克隆主项目仓库
git clone https://github.com/MingLLuo/TIMKE.git

# 克隆OW-ChCCA-KEM实现(如需使用)
git clone https://github.com/MingLLuo/OW-ChCCA-KEM.git
\end{minted}

\subsubsection{安装依赖}

项目使用Go模块管理依赖,自动下载所需库:

\begin{minted}[breaklines]{bash}
cd TIMKE
go mod download
\end{minted}

\subsubsection{编译项目}

编译客户端和服务器程序:

\begin{minted}[breaklines]{bash}
# 编译服务器
go build -o timke-server ./cmd/server

# 编译客户端
go build -o timke-client ./cmd/client

# 编译性能测试工具(可选)
go build -o kem-bench ./cmd/kemBench
go build -o protocol-bench ./cmd/protocolBench
\end{minted}

\subsubsection{配置环境变量(可选)}

为便于使用,可将程序路径添加到系统PATH:

\begin{minted}[breaklines]{bash}
# Linux/macOS
export PATH=$PATH:$(pwd)

# 永久添加(添加到~/.bashrc或~/.zshrc)
echo 'export PATH=$PATH:INSTALLATION_PATH' >> ~/.bashrc
\end{minted}

安装完成后,可通过项目提供的演示脚本体验完整功能:

\begin{minted}[breaklines]{bash}
# 运行演示脚本
cd TIMKE
./scripts/demo.bash
\end{minted}
通过以上步骤,您可以在本地环境中成功安装和运行TIMKE协议实现。