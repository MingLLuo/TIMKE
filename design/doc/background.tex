\section{课题背景及设计目标}

\subsection{课题背景}

随着网络应用的日益普及,安全通信成为现代网络基础设施的核心需求。密钥交换协议允许通信双方在不安全的网络环境中建立共享密钥,为后续的安全通信提供基础。传统密钥交换协议如Diffie-Hellman和RSA在经典计算环境下已被广泛采用,但随着量子计算技术的迅速发展,这些依赖于离散对数和大整数分解难题的密码学原语面临着严峻挑战。

美国国家标准与技术研究院(NIST)于2016年启动后量子密码标准化竞赛,并于2022年公布了首批后量子密码标准算法。其中,基于格的密码学因其安全性依赖于格问题的计算困难性,被认为能够抵抗量子计算攻击,成为后量子密码学研究的重点方向。

传统的密钥交换协议通常将密钥建立与后续安全通信视为分离的阶段。然而,现代网络协议如TLS 1.3、QUIC和Signal等为了提高效率,往往将这些阶段交织在一起,通过多轮协商导出最终密钥。多阶段密钥交换(Multi-stage Key Exchange, MSKE)模型正是为了描述这种复杂交互过程而提出的。MSKE协议将整个交互划分为多个独立阶段,不同阶段可以采用不同的密钥协商与派生方式,生成的阶段密钥可用于不同目的,如消息加密、认证或完整性保护等。

基于密钥封装机制(Key Encapsulation Mechanism, KEM)的密钥交换协议在后量子安全性、高效性和设计简洁性方面具有显著优势。与传统公钥加密(PKE)相比,KEM更适合与混合加密方案结合使用,为现代密码协议提供标准化框架。NIST后量子密码标准中的多个算法都采用KEM设计范式,如基于模格的密钥封装(ML-KEM,前身为Kyber)。

为应对这一技术演进趋势,本文实现了一种紧致安全多阶段密钥交换协议(TIghtly secure Multi-stage Key Exchange,TIMKE),基于后量子安全的KEM构建,特别关注紧致安全性、后量子安全性和实际应用需求。

\subsection{设计目标}

本设计的主要目标是实现一个基于KEM的两阶段紧致安全多阶段密钥交换协议,满足现代网络应用对安全性、效率和灵活性的要求。具体设计目标如下:

\subsubsection{安全目标}

\begin{itemize}
    \item \textbf{后量子安全性}:协议采用基于格的密码学原语,其安全性依赖于在量子计算模型下仍被认为困难的格问题,确保在量子计算时代保持安全性。
    
    \item \textbf{紧致安全性}:协议安全性证明中的归约损失不随会话数量线性增长,确保在大规模部署环境中维持高安全边际。
    
    \item \textbf{多阶段安全性}:支持两阶段密钥派生,第二阶段提供弱前向安全性(即使服务器长期密钥泄露,只要对第二阶段仅采取被动攻击,主会话密钥仍能保持安全)。
    
    \item \textbf{单边认证安全性}:仅对服务器进行身份认证,客户端可匿名参与,简化协议设计并满足普通Web访问等场景的安全需求。
\end{itemize}

\subsubsection{功能目标}

\begin{itemize}
    \item \textbf{双阶段密钥派生}:实现两阶段密钥派生机制,第一阶段基于OW-ChCCA安全的KEM,第二阶段基于OW-PCA安全的KEM,满足不同安全需求。
    
    \item \textbf{0-RTT数据传输}:支持零往返时间数据传输,允许客户端在首个消息中直接发送加密应用数据,显著减少连接建立延迟。
    
    \item \textbf{通用构造与灵活配置}:支持集成不同的后量子安全KEM算法,如ML-KEM、自定义OW-ChCCA KEM等,提供参数优化选项适应不同使用场景。
    
    \item \textbf{实用性能平衡}:在保证安全性的前提下,优化计算效率和通信开销,使协议在普通计算设备上可行运行。
\end{itemize}

\subsubsection{技术要求}

\begin{itemize}
    \item \textbf{模块化设计}:采用严格的模块化架构,使各组件可独立更新和维护,提高代码可读性和可维护性。
    
    \item \textbf{接口标准化}:设计统一的KEM接口,支持不同KEM算法的无缝集成,确保协议在不同环境中的可适配性。
    
    \item \textbf{全面测试验证}:构建完整的测试套件,包括单元测试、集成测试和性能测试,验证协议在各种条件下的正确性和安全性。
    
    \item \textbf{开源可用}:提供可重用的开源实现,支持社区审查和改进,促进后量子安全通信技术的推广。
\end{itemize}

通过实现这些设计目标,本项目旨在为后量子安全通信提供一个理论上严格、实践中可行的解决方案,填补目前在多阶段密钥交换协议实际实现方面的空白。