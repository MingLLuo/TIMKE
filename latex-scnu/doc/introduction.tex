\section{引言}

\subsection{选题背景与意义}

密钥交换协议允许参与方协商建立共享密钥,进而使用安全通道协议为报文交换提供保密性与完整性。认证密钥交换协议在此基础上附加关于身份信息的认证,有效解决传统协议易受中间人攻击的问题。然而,这些协议并未充分考虑现实中复杂的密钥交换。现代网络的背景下,密钥交换和通道建立的两个阶段往往相互交织,参与方通过协商导出最终的密钥,例如 QUIC、 TLS1.3 和 Signal。这种交互过程可由多阶段密钥交换(Multi-stage Key Exchange, MSKE)协议描述。MSKE 协议将整个交互过程划分出多个独立阶段,不同阶段可以采用不同的密钥进行协商与派生,生成的阶段密钥可用于不同的目的,如消息加密、认证或完整性保护等。

美国国家标准与技术研究院(NIST)于2016年启动后量子密码标准化竞赛,以应对量子计算带来的挑战。2022年7月,NIST公布了首批后量子密码标准算法。基于格的密码学(Lattice-based Cryptography)方案在量子计算环境下仍保持较高安全强度,是现代后量子密码学研究的重点方向。目前广泛使用的QUIC、TLS 1.3等网络协议尚未将这类后量子安全算法正式嵌入协议并标准化,突出了研究并实现具有后量子安全特性的MSKE协议的重要意义。

论文\cite{timke_2024}中提出基于密钥封装机制(Key-Encapsulation Mechanism, KEM)的两阶段紧致安全多阶段密钥交换协议(TIghtly secure Multi-stage Key Exchange,记为TIMKE)。TIMKE协议采用两种不同安全性的KEM作为构建基础,第一阶段使用单向可检测选择密文安全KEM提供服务器认证,第二阶段使用单向明文可检测安全KEM建立具有弱前向安全性的会话密钥。TIMKE协议支持零往返时间(Zero Round Trip Time, 0-RTT)数据传输,适用于低延迟通信场景,同时提供后量子安全保障,实现该协议能为后量子安全通讯领域提供解决方案。

\subsection{国内外研究现状和相关工作}

目前,国内外对 MSKE 协议的研究主要集中在协议设计、安全性分析和安全模型构建等方面。Fischlin 等学者\cite{fischlin_multi-stage_2014}提出了一种 MSKE 安全模型,并在此模型基础上对 QUIC 协议的安全性进行了分析与证明。此后,研究者们使用类似方法对多种 MSKE 协议的安全性进行了系统性分析。 Cohn-Gordon 等学者\cite{cohn-gordon_formal_2016}将 MSKE 模型推广至更新的 Signal 协议, 并对 Signal 协议下的密钥协商与密钥派生概率进行了分析; Dowling 等学者\cite{dowling_cryptographic_2020}使用 MSKE 模型来对 TLS 1.3 协议进行安全分析。在他们的研究中,TLS 1.3 握手过程中派生的会话密钥被赋予了多种安全属性标记,包括认证类型(未认证、单边认证或双边认证)、前向安全性保障以及抵抗重放攻击的能力等; Schwabe 等学者\cite{schwabe_post-quantum_2020}提出了首个基于 KEM 的 TLS 协议,在 MSKE 安全模型下证明其通用构造的安全性,同时还提供了一种后量子安全的协议框架。陈霄和王宝成\cite{ideal_lattice_2023}基于MSKE模型提出了一种方案,通过预共享的口令进行认证,并使用 Peikert 误差消除机制结合服务器静态密钥实现多阶段密钥协商,实现了双向认证和二阶会话密钥完美前向保密等特性。

紧致性是密码协议安全性的评估标准之一。安全规约作为面向密码学协议的安全性证明技巧,通过定义敌手能获得的攻击优势,将其与已知的计算性难题联系起来。传统证明的规约损失因子会直接影响实际部署时的安全参数选择,较大的安全损失迫使实现者采用更大的密钥长度和参数,大幅提高计算和通信开销。紧致安全协议能将规约损失控制在可接受范围内,使协议在维持高安全性的同时保持高效运行。业界已经提出多种具有紧致安全性的密钥交换协议。Bader 和 Jager 等人\cite{bader_tightly-secure_2014}构建了首个紧致安全的认证密钥交换协议,其安全性不会随着用户数量或会话数量的增加而降低,并在增强版 Bellare-Rogaway 安全模型中证明了其紧致安全性。针对更复杂的 MSKE 协议,Diemert 和 Jager \cite{davis_concrete_2022}在随机预言模型(Random Oracle, RO)中紧密地将 TLS 1.3 的安全性归约到其密码学组件的多用户安全性,指出通过替换特定组件可获得完全紧致安全的TLS协议,但该方案尚不满足后量子安全要求。

KEM在后量子安全性、高效性、设计简洁性以及与混合加密的兼容性方面优于传统的公钥加密(Public Key Encryption,PKE)。现代加密通常采用混合方法,即用公钥加密短的对称密钥,后使用对称密钥加密长消息,KEM 范式为这种混合加密方式提供了标准框架。 设计和分析安全的 KEM 比设计安全的 PKE 更容易,这也是其被现代密码协议广泛采用的原因之一。

在TIMKE协议相关研究方面,Pan等学者\cite{pan_lattice-based_2023}详细阐述了基于格的单向可检测选择密文安全(One-Way Checkable security against Chosen-Ciphertext Attacks,OW-ChCCA)KEM构造,为紧致安全的密钥交换协议提供了理论基础,但缺乏具体实现与性能分析。

2024 年 8 月,NIST 发布了基于模格的密钥封装(Module-Lattice-Based Key-Encapsulation Mechanism, ML-KEM)标准\cite{nist_mlkem_2024}。该标准的安全性基于模上带误差学习(Module-Learning With Errors, MLWE)问题,这是Regev于2005年提出的带误差学习(Learning With Errors, LWE)问题的扩展。ML-KEM 首先基于MLWE构造公钥加密方案,通过 Fujisaki-Okamoto 变换将其转换为满足“选择密文攻击下的不可区分”(INDistinguishability under adaptive Chosen Ciphertext Attack, IND-CCA2) 的KEM方案,如Google、Cloudflare等公司已在运行实例中部署了基于ML-KEM的混合方案\cite{noauthor_pqc_2025},兼容传统和后量子安全需求。

\subsection{设计内容与贡献}

通过分析TIMKE协议,将论文\cite{timke_2024}中的构造进行实例化,验证其现实可行性。TIMKE的实现需要调用 OW-ChCCA KEM 实例,现实中并没有对应的实现。基于论文\cite{pan_lattice-based_2023}中的OW-ChCCA KEM构造方案构建出一套OW-ChCCA KEM库,并评估其实际性能。协议实现统一的KEM接口,用户可以通过简单封装嵌入不同KEM实例,比如Golang标准库下的ML-KEM,为协议提供多种安全与性能配置选项。实现构建了一套完整的客户端与服务器演示系统,支持0-RTT数据传输。测试模块提供全面的性能评估框架,分析不同KEM组合在TIMKE协议中的性能。

