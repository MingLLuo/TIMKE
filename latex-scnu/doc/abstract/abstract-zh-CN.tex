\setcounter{page}{1}
\pagenumbering{Roman}
\begin{center}
  {\zihao{-2}\heiti\bfseries 一种多阶段密钥交换协议的设计与实现}
  \\ \hspace*{\fill}
  \\ \hspace*{\fill} \\
  \addcontentsline{toc}{section}{\zihao{4} 摘\quad 要}
  \zihao{-2} \bfseries 摘\quad 要
\end{center}
  \bigskip
\begin{spacing}{1.5}  
  \zihao{-4}
  本文实现了一种基于密钥封装机制的紧致安全多阶段密钥交换(TIghtly secure Multi-stage Key Exchange,TIMKE)协议,基于后量子安全原语构建,现阶段能抵抗量子计算攻击,其“紧致安全”的特性确保协议安全性不随用户数量或会话数量增加而降低。TIMKE采用两阶段设计:第一阶段基于服务器长期密钥提供身份认证并支持零往返时间(Zero Round Trip Time, 0-RTT)数据传输;第二阶段结合临时密钥提供弱前向安全保障。
  我们实现了基于格的单向可检测选择密文安全(One-Way Checkable security against Chosen-Ciphertext Attacks,OW-ChCCA)密钥封装机制,通过参数优化使其能在常规计算环境中运行,同时集成了基于模格的密钥封装机制(Module-Lattice based Key-Encapsulation Mechanism, ML-KEM)作为高性能替代方案。实现采用模块化架构,支持灵活组合不同密码原语,满足各种安全需求和性能约束。
  实验结果表明,虽然OW-ChCCA KEM的实现在计算资源方面要求较高,但基于ML-KEM的TIMKE变体能在毫秒级时间内完成密钥交换,验证了该协议在实际环境中的可行性。
\end{spacing}
    \medskip
  \zihao{-4}
  {\heiti \bfseries 关键词:}\ \ 多阶段密钥交换协议;后量子安全;协议实例化
\newpage