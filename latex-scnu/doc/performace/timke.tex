\subsection{TIMKE性能评估}

本节对不同配置下的TIMKE协议进行性能评估,分析各核心组件对整体性能的影响,并探讨保持安全性下的优化策略,为实际部署决策提供参考。

\subsubsection{评估方法}

实现提供一套性能评估框架,对协议各组件及其组合进行系统测试,包括:1)组件级基准测试,重点评估各KEM实现的关键操作(密钥生成、封装和解封装)性能,识别潜在瓶颈;2)协议阶段评估,分别测量第一阶段(包含0-RTT)和第二阶段的执行时间与资源消耗;3)端到端性能分析,评估完整协议流程的时间效率、内存占用和通信开销,模拟真实应用场景。替代实现对比则测试不同KEM组合的性能差异,分析最佳配置方案。

所有测试在统一的环境中进行,硬件平台为Apple M1 Pro (8核心CPU,16GB RAM),操作系统为MacOS Sequoia 15,开发语言为Go 1.23.4。每项测试重复10次取平均值,消除随机波动。测试工具集成于协议代码库,确保测量的准确性与一致性,不同身份的参与方会占用同一设备不同的端口进行交互。
重点测试两套KEM组合配置方案:1)理论构造配置使用OW-ChCCA KEM作为KEM\textsubscript{1}和ML-KEM作为KEM\textsubscript{2},代表原始TIMKE协议设计;2)实用替代配置使用ML-KEM同时作为KEM\textsubscript{1}和KEM\textsubscript{2},探索在当前技术条件下的最佳性能方案。

\subsubsection{混合KEM实现性能分析}

考虑到OW-ChCCA KEM的性能受限,首先评估折中方案:使用高度简化的OW-ChCCA变体(OWChCCA-mini)作为KEM\textsubscript{1},结合ML-KEM作为KEM\textsubscript{2}。OWChCCA-mini使用极小矩阵维度(n=16, m=8192, $\lambda$=16)和高度优化的算法实现。该配置仅作为概念验证使用,不应在实际安全系统中部署,仅提供理论构造可行性的基准数据。

\begin{table}[ht]
\centering
\caption{混合KEM实现的TIMKE协议性能(单位:毫秒)}
\label{tab:timke-hybrid-perf}
\begin{tabular}{|l|r|r|r|r|}
\hline
\textbf{KEM组合} & \textbf{阶段1} & \textbf{阶段2} & \textbf{总时间} & \textbf{内存(KB)} \\
\hline
OWChCCA-mini + ML-KEM-512    & 322.18  & 118.53  & 440.71  & 2,206,372   \\
\hline
OWChCCA-mini + ML-KEM-768    & 324.47  & 117.61  & 442.07  & 2,206,250   \\
\hline
OWChCCA-mini + ML-KEM-1024   & 374.76  & 127.47  & 502.23  & 2,204,554   \\
\hline
\end{tabular}
\end{table}

即使采用极度简化的OW-ChCCA变体,混合方案的性能仍然受到限制,不同配置总执行时间分别为440.71ms、442.07ms和502.23ms,在不稳定的网络环境下有更明显的延迟感知。第一阶段(使用OWChCCA-mini)执行时间占总时间的73\%至75\%,明确指出性能瓶颈所在。在高安全级别下更为明显,如OWChCCA-mini与ML-KEM-1024组合的配置中,第一阶段耗时达到374.76ms,几乎是第二阶段的3倍。所有混合配置均需要超过2GB的内存,远超移动设备和嵌入式系统的可用资源。较差的性能表现主要源于OWChCCA-mini中的矩阵运算,尽管较大程度的简化参数设置,但基本的LWE问题结构仍然导致过多的内存开销。

数据表明,尽管混合KEM方案在理论上实现了TIMKE协议的设计目标,但其实际性能表现使其难以直接应用于对延迟敏感的现代通信系统。

\subsubsection{ML-KEM替代实现性能分析}

本小节将分析使用ML-KEM同时替代KEM\textsubscript{1}和KEM\textsubscript{2}的方案。这种配置在理论上可能不提供与原设计相同的紧致安全保证,但为评估TIMKE协议在实际环境中的可行性提供了参考。

表\ref{tab:timke-mlkem-perf}展示了使用ML-KEM不同安全级别实现的TIMKE协议性能。

\begin{table}[ht]
\centering
\caption{基于ML-KEM的TIMKE协议性能(单位:毫秒)}
\label{tab:timke-mlkem-perf}
\begin{tabular}{|l|r|r|r|r|}
\hline
\textbf{KEM组合} & \textbf{阶段1} & \textbf{阶段2} & \textbf{总时间} & \textbf{内存(KB)} \\
\hline
ML-KEM-512 + ML-KEM-512    & 0.10  & 0.07  & 0.17  & 26   \\
\hline
ML-KEM-768 + ML-KEM-768    & 0.15  & 0.10  & 0.25  & 40   \\
\hline
ML-KEM-1024 + ML-KEM-1024  & 0.21  & 0.14  & 0.35  & 59   \\
\hline
\end{tabular}
\end{table}

在统一测试基准下,基于ML-KEM的配置比混合KEM方案快约3000倍。即使是安全级别最高的ML-KEM-1024配置,总执行时间也仅为0.35毫秒,完全满足高性能网络应用的需求。如此大的差距主要因为ML-KEM高效的算法结构和实现上的优化。ML-KEM基于模格结构,能够利用数论变换(NTT)等方式加速计算,其参数设置也经过精心优化,在保持安全的同时最小化计算开销。

内存使用方面,ML-KEM方案表现优秀。测试配置的内存消耗分别为26KB、40KB和59KB,比混合方案低4-5个数量级,使TIMKE协议能够在资源受限的环境中运行,如移动设备、物联网终端甚至某些嵌入式系统。低内存占用也意味着更低的能耗和更好的并发性能,对服务器端的部署尤为重要。

通信开销方面,基于ML-KEM的配置表现出色。ClientHello消息大小约为2-3KB,ServerResponse约为1-2KB,总通信量控制在5KB以内。阶段时间分布也更加均衡,第一阶段和第二阶段分别占总时间的约60\%和40\%。平均的时间分布在保证了0-RTT数据传输的低延迟特性的同时,不为后续阶段引入较大的额外延迟。

\paragraph{安全性降级影响分析}
从安全模型角度,OW-ChCCA安全专为紧致安全多挑战多用户场景设计,其安全损失与用户数量$n$和会话数量$q_s$无关,安全规约损失为常数$\mathcal{O}(1)$。ML-KEM虽然提供了IND-CCA2安全,但在多用户环境下其安全损失会随用户数量增加,典型的安全损失为$\mathcal{O}(n \cdot q_s)$。
对于小规模系统(如用户数$n < 10^3$,每用户会话数$q_s < 10^3$),ML-KEM配置下的安全性损失可接受,实际参数选择与理论安全级别差距可控;对于中等规模系统(如用户数$n \approx 10^6$,每用户会话数$q_s \approx 10^3$),安全损失达到$\mathcal{O}(10^9)$量级,需要增加密钥长度(约30比特)才能维持原有安全级别;对于大规模系统(如用户数$n > 10^8$,每用户会话数$q_s > 10^4$),安全损失将超过$\mathcal{O}(10^{12})$量级,安全参数增加需求可能使方案在计算和通信效率上不再具有实用性。

使用ML-KEM替代OW-ChCCA KEM作为KEM\textsubscript{1}会对TIMKE协议的整体安全性产生影响,具体如下:

\begin{enumerate}
    \item \textbf{紧致安全性}:完全失去紧致安全保证,安全损失会随系统规模增长。
    
    \item \textbf{后量子安全性}:保持不变,ML-KEM本身具有抗量子计算攻击能力。
    
    \item \textbf{0-RTT安全性}:基本保持,但在大规模部署中,可能需要增加密钥长度以补偿安全损失。
    
    \item \textbf{弱前向安全性}:第二阶段的弱前向安全性基本保持,因为它主要依赖于KEM\textsubscript{2}的安全性。
\end{enumerate}

综合来看,ML-KEM替代方案在部署TIMKE协议的中小规模系统中可以接受,安全降级影响有限。大规模部署环境则需权衡效率与安全性,可能要求更高安全级别的参数以补偿紧致安全性的损失。