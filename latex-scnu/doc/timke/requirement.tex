\subsection{设计目标与实现策略}

实现TIMKE协议前需要明确设计目标并制定相应的实现策略,协议的安全性建立在理论模型之上,部署实际系统还需要考虑运行效率、资源占用、可维护性以及兼容性等多方面因素。

现代密码协议要求支持长期维护,因此TIMKE协议的实现需要采用模块化架构设计,将KEM操作、消息序列化、对称加密等功能封装为独立模块,组件间通过定义明确的接口交互(如图\ref{fig:system-architecture}所示),允许在保持协议核心逻辑不变的情况下,替换不同KEM实现或更新协议组件,提高系统可维护性和可扩展性。实现兼容NIST后量子密码标准,确保协议长期可用性。

协议还需支持后续信息交换和会话管理,TIMKE协议的实现包含简洁而功能完备的报文交换框架,其序列化格式和接口设计参考现代网络协议(如TLS 1.3),采用ClientHello和ServerResponse消息结构。后量子安全算法通常比传统密码算法需要更多计算资源和通信带宽,为适应不同使用场景,实现提供参数优化选项,针对资源受限环境(如移动设备和IoT设备)提供不同安全级别的配置选择,单元测试、集成测试等性能测试套件提供为协议部署参考。实现还提供一套支持0-RTT特性的报文传输演示系统,提供会话密钥更新和会话终止等基本会话管理功能。
