\subsection{协议流程与工作原理}

TIMKE协议工作流程如图\ref{fig:protocol-flow}所示,分为三个主要阶段:预共享阶段、临时密钥协商与0-RTT传输阶段,以及主会话密钥建立阶段。

\subsubsection{预共享阶段}
在正式协议交互开始前,服务器需要预先生成长期密钥对并将公钥分发给客户端:

\begin{enumerate}
    \item 服务器S生成KEM\textsubscript{1}的密钥对:$(pk_S, sk_S) \leftarrow \text{KEM}_1.\text{Gen}(par)$
    \item 服务器将公钥$pk_S$通过可信渠道或证书分发给客户端C
\end{enumerate}

这一阶段类似TLS协议中的证书预分发,确保客户端能够获取并验证服务器的公钥,预共享阶段的安全性依赖于公钥分发机制的安全性,如证书机构的可信度。

\subsubsection{第一阶段:临时密钥协商与 0-RTT 传输}

第一阶段的主要目标是快速建立初始会话密钥,支持0-RTT数据传输。该阶段的交互效率直接影响用户感知的连接延迟,具体流程如下:

\begin{enumerate}
    \item \textbf{客户端C}执行以下操作:
    \begin{itemize}
        \item 生成KEM\textsubscript{2}的临时密钥对:$(epk_C, esk_C) \leftarrow \text{KEM}_2.\text{Gen}(par)$
        \item 使用服务器公钥封装共享密钥:$(C_1, K_1) \leftarrow \text{KEM}_1.\text{Encap}(pk_S)$
        \item 派生临时会话密钥:$K_{tmp} := H_1(pk_S, C_1, K_1)$
        \item 加密0-RTT数据:$C_{payload} \leftarrow \text{Enc}(K_{tmp}; M_{0-RTT})$
        \item 发送消息$(epk_C, C_1, C_{payload})$给服务器S
    \end{itemize}

    \item \textbf{服务器S}收到消息后执行以下操作:
    \begin{itemize}
        \item 解封装得到共享密钥:$K_1 := \text{KEM}_1.\text{Decap}(sk_S, C_1)$
        \item 派生相同的临时会话密钥:$K_{tmp} := H_1(pk_S, C_1, K_1)$
        \item 解密0-RTT数据:$M_{0-RTT} := \text{Dec}(K_{tmp}, C_{payload})$
    \end{itemize}
\end{enumerate}

第一阶段结束后,双方已经建立了临时会话密钥$K_{tmp}$,客户端能够在第一个往返时间内发送加密的应用数据,实现0-RTT特性。对需要立即发送数据的时延敏感场景(如Web页面加载、API调用)尤为重要,能减少用户感知的延迟。

\subsubsection{第二阶段:主会话密钥建立}

第二阶段的目标是建立更安全的主会话密钥,该密钥具有弱前向安全性。尽管第一阶段已经建立了安全通道,但第二阶段通过结合双方的密钥材料进一步增强安全性:
\begin{enumerate}
    \item \textbf{服务器S}继续执行以下操作:
    \begin{itemize}
        \item 使用客户端临时公钥封装共享密钥:$(C_2, K_2) \leftarrow \text{KEM}_2.\text{Encap}(epk_C)$
        \item 派生主会话密钥:$K_{main} := H_2(pk_S, epk_C, C_1, C_2, K_1, K_2)$
        \item 加密应用数据:$C'_{payload} \leftarrow \text{Enc}(K_{main}; M_1)$
        \item 发送消息$(C_2, C'_{payload})$给客户端C
    \end{itemize}

    \item \textbf{客户端C}收到消息后执行以下操作:
    \begin{itemize}
        \item 使用临时私钥解封装共享密钥:$K_2 := \text{KEM}_2.\text{Decap}(esk_C, C_2)$
        \item 派生相同的主会话密钥:$K_{main} := H_2(pk_S, epk_C, C_1, C_2, K_1, K_2)$
        \item 解密服务器发送的数据:$M_1 := \text{Dec}(K_{main}, C'_{payload})$
    \end{itemize}
\end{enumerate}

第二阶段结束后,双方建立了具有更强安全性的主会话密钥$K_{main}$,该密钥结合双方的共享密钥信息,具有Stage-2弱前向安全性。与第一阶段相比,即使服务器的长期密钥泄露,只要攻击者不主动干预通信过程,主会话密钥仍然安全,为长期会话提供额外的安全保障。

\subsubsection{密钥派生机制}

TIMKE协议的核心是两个密钥派生的步骤,分别用于生成临时会话密钥和主会话密钥,不仅确保密钥的唯一性和随机性,还进行了会话的上下文绑定,防止中间人攻击和会话混淆:

\textbf{临时会话密钥派生.} 临时会话密钥通过以下公式计算:
\begin{equation}
K_{tmp} := H_1(pk_S, C_1, K_1)
\end{equation}

这一派生过程结合服务器公钥$pk_S$、封装密文$C_1$和共享密钥$K_1$三个关键元素,通过密码学哈希函数$H_1$导出临时会话密钥。确保只有持有正确服务器私钥$sk_S$的一方才能计算出相同的$K_{tmp}$,实现隐式身份认证。将服务器公钥包含在派生过程中可以防止中间人攻击,而封装密文的包含则确保了密钥的唯一性。

\textbf{主会话密钥派生.} 主会话密钥通过以下更复杂的公式计算:
\begin{equation}
K_{main} := H_2(pk_S, epk_C, C_1, C_2, K_1, K_2)
\end{equation}

主会话密钥的派生结合了更全面的上下文信息,包括服务器公钥$pk_S$、客户端临时公钥$epk_C$、两个封装密文$C_1$和$C_2$,以及两个共享密钥$K_1$和$K_2$。主会话密钥的安全不仅依赖于服务器的长期密钥,还依赖于临时生成的密钥信息提供的弱前向安全性。即使服务器长期密钥泄露,攻击者仍需要获取临时密钥才能恢复主会话密钥。

实现中两个哈希函数均使用SHA3-512,通过不同的域分隔符确保它们的输出相互独立,以防止不同上下文中使用相同哈希函数导致的安全问题。