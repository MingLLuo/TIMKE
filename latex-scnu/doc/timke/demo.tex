\subsection{演示系统设计与功能}

实现构建了一套完整的客户端-服务器演示系统,支持0-RTT数据传输和端到端加密通信,基于命令行进行交互,包含三个主要组件:服务器程序、客户端程序和演示脚本。

服务器程序负责接收请求、处理密钥协商消息并响应客户端。客户端程序负责启动密钥交换流程、发送0-RTT数据并与服务器进行加密通信。演示脚本提供用户友好的界面,自动化协议执行流程,用户无需了解实现细节即可体验协议的完整流程。为提高系统可用性并支持多样化测试场景,各组件均提供丰富的配置选项,包括服务器地址、端口设置、KEM算法选择、密钥存储位置和日志详细程度等。支持多种运行模式,包括密钥生成、服务器监听、单次连接客户端和交互式客户端模式,满足不同测试和演示需求。所有组件都集成了彩色输出支持,通过不同颜色区分各类信息提升用户体验和问题诊断效率。

完整的演示工作流程如下:

首先,系统生成长期服务器密钥对,并将公钥输出为可供客户端导入的格式,模拟实际部署环境中的预共享阶段,确保客户端能够获取经过身份验证的服务器公钥:

\begin{minted}{bash}
$ ./demo.bash
[选择选项1: 生成服务器密钥]
\end{minted}

系统会生成指定KEM格式的密钥对,存储在配置的位置(默认为.temp目录)。服务器私钥安全存储,公钥则生成为客户端可导入的格式,确保后续通信的认证基础。

接下来,系统启动服务器程序,加载生成的私钥并开始监听指定端口:

\begin{minted}{bash}
$ ./demo.bash
[选择选项2: 启动服务器]
\end{minted}

服务器初始化过程中会输出配置参数,包括使用的KEM算法、公钥摘要和监听地址,并进入等待连接状态。

\paragraph{3. 客户端连接} 
用户可选择两种客户端连接模式:携带0-RTT数据的单次连接,或支持持续交互的交互式模式:

\begin{minted}{bash}
$ ./demo.bash
[选择选项3: 运行带0-RTT数据的客户端]
或
$ ./demo.bash
[选择选项4: 运行交互式客户端]
\end{minted}

在首次连接时,客户端加载服务器公钥,生成临时密钥对,并执行TIMKE协议第一阶段。若选择0-RTT模式,客户端会在初始消息中包含加密的应用数据(默认为"Hello from TIMKE client! This is 0-RTT data.");若选择交互式模式,客户端完成完整的两阶段密钥协商后,进入命令行交互界面,支持输入消息并接收服务器加密响应。

当客户端发起连接时,系统执行完整的TIMKE协议流程,客户端生成ClientHello消息并发送给服务器,服务器处理请求并解封装KEM\textsubscript{1}密文恢复临时会话密钥,随后服务器使用客户端临时公钥生成KEM\textsubscript{2}密文派生主会话密钥并返回ServerResponse,最后客户端处理响应并派生相同的主会话密钥,完成整个协议流程。

在交互式模式下,用户可以通过命令行输入消息,这些消息使用主会话密钥加密后发送给服务器:

\begin{minted}{bash}
> Hello, this is an encrypted message from the client
Server received: Hello, this is an encrypted message from the client
\end{minted}

服务器收到加密消息后,使用相同的会话密钥解密,处理明文内容,并返回加密响应。过程模拟了实际应用环境中的安全通信,验证了会话密钥的正确派生和对称加密的有效性。

演示系统充分考虑用户体验,提供清晰的状态反馈和错误信息,便于用户理解协议流程并追溯潜在问题,还内置资源清理和会话管理功能,确保在异常情况下能够正确释放资源并重置。该系统不仅验证了TIMKE协议的可行性,也为实际应用提供参考。