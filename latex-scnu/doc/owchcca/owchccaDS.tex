\subsection{核心数据结构与算法}
\subsubsection{数据结构设计}

OW-ChCCA-KEM的核心数据结构包括公钥、私钥、密文以及支持矩阵运算的其他变量:

\begin{itemize}
    \item \textbf{公钥结构 (PublicKey)}:
    \begin{align}
    pk = (params, \mathbf{A}, \mathbf{u}_0, \mathbf{u}_1)
    \end{align}
    其中$params$包含方案参数,$\mathbf{A} \in \mathbb{Z}_q^{n \times m}$是随机矩阵,$\mathbf{u}_0, \mathbf{u}_1 \in \mathbb{Z}_q^{n \times \lambda}$是两个特殊矩阵,分别包含真实数据和随机数据。
    
    \item \textbf{私钥结构 (PrivateKey)}:
    \begin{align}
    sk = (pk, \mathbf{Z}_b, b)
    \end{align}
    其中$pk$是对应的公钥引用,$\mathbf{Z}_b \in \mathbb{Z}_q^{m \times \lambda}$是密钥矩阵,$b \in \{0,1\}$是指示位,决定哪个$\mathbf{u}$矩阵包含真实数据。
    
    \item \textbf{密文结构 (Ciphertext)}:
    \begin{align}
    ct = (c_0, c_1, \mathbf{x}, \hat{H}_0, \hat{H}_1)
    \end{align}
    其中$c_0, c_1$各为$\lambda$比特的密钥材料,$\mathbf{x} \in \mathbb{Z}_q^m$是LWE样本,$\hat{H}_0, \hat{H}_1 \in \mathbb{Z}_q^\lambda$是两个辅助向量。
    
    \item \textbf{矩阵与向量类 (Matrix/Vector)}:
    提供模运算支持的大整数矩阵和向量表示,包含并行算法实现。矩阵类支持加法、乘法、转置等基本运算,向量类支持加法、点积、标量乘法等操作,所有运算都在模$q$下进行。
\end{itemize}

数据结构通过精心设计的序列化与反序列化机制支持二进制编码,以便网络传输和持久化存储,序列化格式采用紧凑表示以最小化传输开销。需要指出,共享参数在许多 KEM 的实现中都是预先制定好的,不同安全级别下有不同的设置,指定一个理论方案的现实参数需要严谨且持久的分析测试,在尚未确定具体参数前,实现将共享参数作为公钥的一部分,所占空间与其他元素相比可忽略不计。

\subsubsection{核心算法实现}

OW-ChCCA-KEM包含三个核心算法:密钥生成、密钥封装和密文解封装,确保密钥派生的安全性和正确性。

密钥生成算法(Algorithm \ref{alg:key-gen})负责生成安全的公私钥对。首先创建随机矩阵$\mathbf{A}$并生成密钥矩阵$\mathbf{Z}_b$,随后根据随机选择的比特$b$构建包含真实数据和随机数据的公钥矩阵对$(\mathbf{u}_0, \mathbf{u}_1)$。通过确保攻击者无法判断哪个矩阵包含真实信息,从而保证系统安全性。

\begin{algorithm}
\caption{OW-ChCCA-KEM密钥生成算法}
\label{alg:key-gen}
\begin{algorithmic}[1]
\Procedure{KeyGen}{$params$}
    \State 从均匀分布采样矩阵 $\mathbf{A} \stackrel{\$}{\leftarrow} \mathbb{Z}_q^{n \times m}$
    \State 随机选择比特 $b \stackrel{\$}{\leftarrow} \{0, 1\}$
    \State 从高斯分布采样矩阵 $\mathbf{Z}_b \leftarrow D_{\mathbb{Z}^{m \times \lambda}, \alpha}$,其中 $\alpha = \sqrt{n}$
    \State 计算矩阵乘积 $\mathbf{A}\mathbf{Z}_b \in \mathbb{Z}_q^{n \times \lambda}$
    \If{$b = 1$}
        \State $\mathbf{u}_1 \gets \mathbf{A}\mathbf{Z}_b$
        \State 从均匀分布采样 $\mathbf{u}_0 \stackrel{\$}{\leftarrow} \mathbb{Z}_q^{n \times \lambda}$
    \Else
        \State $\mathbf{u}_0 \gets \mathbf{A}\mathbf{Z}_b$
        \State 从均匀分布采样 $\mathbf{u}_1 \stackrel{\$}{\leftarrow} \mathbb{Z}_q^{n \times \lambda}$
    \EndIf
    \State 构造公钥 $pk \gets (params, \mathbf{A}, \mathbf{u}_0, \mathbf{u}_1)$
    \State 构造私钥 $sk \gets (pk, \mathbf{Z}_b, b)$
    \State \Return $(pk, sk)$
\EndProcedure
\end{algorithmic}
\end{algorithm}

密钥封装算法(Algorithm \ref{alg:encap})由发送方执行,将随机生成的密钥材料封装为密文。首先生成随机种子$r$,通过扩展函数$G$派生出向量$s$、扰动采样参数$\rho$和二元向量$h_0$、$h_1$。基于LWE问题,算法出构造样本$\mathbf{x}$和辅助向量$\hat{H}_0$、$\hat{H}_1$,最终生成密文和共享密钥。

\begin{algorithm}
\caption{OW-ChCCA-KEM封装算法}
\label{alg:encap}
\begin{algorithmic}[1]
\Procedure{Encaps}{$pk$}
    \State 生成随机种子 $r \stackrel{\$}{\leftarrow} \{0,1\}^{\lambda}$
    \State 通过扩展函数计算 $(s, \rho, h_0, h_1) \gets G(r)$
    \State 使用 $\rho$ 作为种子从高斯分布采样 $\mathbf{e} \leftarrow D_{\mathbb{Z}^m, \alpha'}$,其中 $\alpha' = n^{2.5} \cdot m$
    \State $\mathbf{x} \gets \mathbf{A}^T\mathbf{s} + \mathbf{e}$
    \State $\hat{H}_0 \gets \mathbf{u}_0^T\mathbf{s} + h_0 \cdot \lfloor q/2 \rfloor$
    \State $\hat{H}_1 \gets \mathbf{u}_1^T\mathbf{s} + h_1 \cdot \lfloor q/2 \rfloor$
    \State $\hat{K}_0 \gets H(\mathbf{x}, \hat{H}_0, h_0)[:{\lambda}]$
    \State $\hat{K}_1 \gets H(\mathbf{x}, \hat{H}_1, h_1)[:{\lambda}]$
    \State $c_0 \gets \hat{K}_0 \oplus r$
    \State $c_1 \gets \hat{K}_1 \oplus r$
    \State 构造密文 $ct \gets (c_0, c_1, \mathbf{x}, \hat{H}_0, \hat{H}_1)$
    \State 应用密钥派生函数计算共享密钥 $K \gets \text{KDF}(r)$
    \State \Return $(ct, K)$
\EndProcedure
\end{algorithmic}
\end{algorithm}

密文解封装算法(Algorithm \ref{alg:decap})由接收方执行,通过私钥验证密文并恢复共享密钥。首先基于私钥中的标志位$b$确定要处理的密文部分,后利用密钥矩阵$\mathbf{Z}_b$进行解码。关键步骤是恢复随机种子$r$并对密文进行验证,不仅检验了密文的完整性和有效性,还隐式地实现了身份认证。

\begin{algorithm}
\caption{OW-ChCCA-KEM解封装算法}
\label{alg:decap}
\begin{algorithmic}[1]
\Procedure{Decaps}{$sk, ct$}
    \State 解析密文 $ct$ 为 $(c_0, c_1, \mathbf{x}, \hat{H}_0, \hat{H}_1)$
    \State 从私钥 $sk$ 提取 $(\mathbf{Z}_b, b)$
    \If{$b = 1$}
        \State 选择 $\hat{H}_b \gets \hat{H}_1$ 和 $c_b \gets c_1$
    \Else
        \State 选择 $\hat{H}_b \gets \hat{H}_0$ 和 $c_b \gets c_0$
    \EndIf
    \State 计算 $\mathbf{Z}_b^T\mathbf{x}$
    \State 计算差值 $\Delta \gets \hat{H}_b - \mathbf{Z}_b^T\mathbf{x}$
    \State 对差值向量舍入获得 $h_b'$
    \State 计算 $\hat{K}_b \gets H(\mathbf{x}, \hat{H}_b, h_b')[:{\lambda/8}]$
    \State 恢复 $r \gets c_b \oplus \hat{K}_b$
    \State 通过扩展函数计算 $(s', \rho', h_0', h_1') \gets G(r)$
    \State 使用 $\rho'$ 重新采样 $\mathbf{e}'$ 并验证 $\mathbf{x} = \mathbf{A}^T\mathbf{s}' + \mathbf{e}'$
    \State 验证 $\hat{H}_0 = \mathbf{u}_0^T\mathbf{s}' + h_0' \cdot \lfloor q/2 \rfloor$ 和 $\hat{H}_1 = \mathbf{u}_1^T\mathbf{s}' + h_1' \cdot \lfloor q/2 \rfloor$
    \If{验证通过}
        \State 应用密钥派生函数计算共享密钥 $K \gets \text{KDF}(r)$
        \State \Return $K$
    \Else
        \State \Return $\perp$ (解封装失败)
    \EndIf
\EndProcedure
\end{algorithmic}
\end{algorithm}